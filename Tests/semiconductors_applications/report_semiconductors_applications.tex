\documentclass[12pt]{article}

\usepackage{lipsum}
\usepackage{mathtools}
\usepackage{titlesec}
\usepackage{fancyhdr}

\title{Applications of Semi-conductors}
\author{Omar Sherif El-Tabarany}
\date{}

\pagestyle{fancy}
\fancyhf{}
\fancyhfoffset[L]{1cm} % left extra length
\fancyhfoffset[R]{1cm} % right extra length
\rhead{Year 1 - Sec 2}
\lhead{\bfseries Omar Sherif El-Tabarany}
\rfoot{}

\begin{document}
	
	\maketitle
	
	\begin{abstract}
		This report explores the diverse applications of semiconductors, highlighting their crucial role in modern technology, including fields such as electronics, telecommunications, renewable energy, and healthcare. It delves into the multifaceted applications of semiconductors, elucidating their significance across industries like information technology, and consumer electronics. It also addresses emerging trends and potential future developments in semiconductor applications, emphasizing their impact on advancing technological landscapes.
	\end{abstract}	
		
	\tableofcontents
	
	\section{Information Technology (IT) Integration}
	Semiconductors are pivotal to modern computing technology. They enable the production of microprocessors and memory chips, which are the primary components in computers, servers, and data centers. These devices are used across various industries, from finance and healthcare to manufacturing and logistics. (1) Semiconductors are essential for data processing and storage. They facilitate the development of high-performance memory systems, driving the development of increasingly complex semiconductors that can facilitate intense memory operations. (2) Semiconductors are the base for most electronics, enabling the construction of complex integrated circuits that power advanced technologies for healthcare, communications, computing, and transportation, among other applications. (3)
	
	\section{Consumer Electronics Evolution}
	The evolution of consumer electronics, including smart devices and wearables, has been significantly influenced by semiconductor advancements. Semiconductors have played a crucial role in driving the development of IoT, smart devices, and 5G technology, leading to faster advancements in AI and computing power. It is estimated that semiconductors account for about 50\% of computing hardware costs, and they are essential for integrating AI computing devices into society seamlessly and pervasively.(4) Semiconductors have become an essential part of consumer electronics, including televisions, smartphones, refrigerators, dishwashers, washing machines, airplanes, and automobiles. The demand for more powerful, lighter, and smaller semiconductors has been driven by technological applications, impacting various aspects of everyday life, from efficient housekeeping to safer travel and more attractive entertainment. The integration of IoT and AI has led to the incorporation of wearable technology into various scenarios, including healthcare, navigation systems, consumer goods, and professional sports. (5)
	
	\section{Challenges and Solutions}
	The need to streamline the semiconductor-industry supply chain is evident. At present, it can take up to six months to complete production of an integrated circuit, not counting packaging and delivery of chips to the buyer. Semiconductor companies need smoother, more efficient manufacturing processes. Automation offers potential solutions. However, automating semiconductor-manufacturing processes isn’t without its own challenges. (6) It’s not a race to the bottom but more of a race to the smallest. The industry is on a relentless pursuit of miniaturization which is considered to be the main driving force behind the industry’s remarkable progress and growth. Moore’s Law, which states that the number of transistors on a chip doubles approximately every two years highlights the pressure to constantly innovate and ensure quality control, yield improvement, and time-to-market. This of course has to be delicately balanced between pushing technological boundaries and ensuring cost-effective and reliable production. (7) While the answer may vary based on a company’s strengths and weaknesses, all semiconductor companies could benefit by rethinking their approach in six critical areas: technology leadership, long-term R\&D, resilience, talent, ecosystem capabilities, and greater capacity. The final area here will not provide immediate benefits, of course, but it could be an important part of a long-term strategy. We have quantified the costs associated with fabs of different sizes to help semiconductor companies determine if capacity expansion is right for them. (8)
	
	\section{Future Trends in Semiconductor Applications}
	Quantum computing is poised to revolutionize the processing of large amounts of data, with potential applications in various industries such as healthcare, finance, communications, cybersecurity, and energy. It is expected to impact physics, chemistry, mathematics, and biology, offering significant computational benefits and applications across multiple disciplines. (9) Addressing the skills shortage, attracting new talent, and supporting the emergence of a skilled workforce are crucial for the future of the semiconductor industry. Developing an in-depth understanding of the global semiconductor supply chains and strengthening research and technology leadership are also essential for future advancements. (10)
	
	\pagebreak
	
	\section{References \& Citations}
		\begin{enumerate}
			\item corporatetraining.usf.edu - What is Semiconductor Technology and Why Is It Important? - Haley DeLeon
			\item irds.ieee.org - New Semiconductor Technologies and Applications
			\item www.nist.gov - semiconductors
			\item www.renesas.com - Semiconductor process technology; History, trends and evolution - Markus Vomfelde
			\item www.acldigital.com - Latest Technological Developments in Embedded and Semiconductor Industry
			\item irds.ieee.org - New Challenges Facing Semiconductors
			\item www.eidasolutions.com - The Semiconductor Industry’s 3 Biggest Challenges
			\item www.mckinsey.com - Strategies to lead in the semiconductor world - Ondrej Burkacky, Marc de Jong, and Julia Dragon
			\item www.forbes.com - The Emerging Paths Of Quantum Computing - Chuck Brooks
			\item www.capgemini.com - 7 major trends shaping the future of the semiconductor industry
		\end{enumerate}
	
\end{document}